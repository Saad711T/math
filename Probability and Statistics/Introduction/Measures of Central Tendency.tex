\documentclass[12pt]{article}

\usepackage{amsmath}  % For mathematical symbols and equations
\usepackage{amssymb}  % For additional math symbols
\usepackage{geometry} % To control page margins
\geometry{a4paper, margin=1in}

\usepackage{hyperref} % To include hyperlinks
\usepackage{fancyhdr} % To customize the header and footer
\pagestyle{fancy}
\fancyhf{}
\fancyhead[L]{Probability and Statistics} 
\fancyfoot[C]{\thepage}

\title{Measures of Central Tendency}
\author{Saad Almalki}
\date{\today}

\begin{document}

\maketitle

\section{Introduction} 
Measures of location, or central tendency, are important concepts in probability and statistics. 

Some of these measures of central tendency are: the mean, median, and mode.

Measures of dispersion are another classification related to measures of central tendency.

\section{Mean} 
The sample mean is the sum of all the numbers in the series divided by the number of elements in the series.\\
\textbf{Example:}\\
$1, 2, 3, 4, 5$ \\
The sample mean is: 
\[
\frac{1+2+3+4+5}{5} = \frac{15}{5} = 3
\]

\section{Median}
The median is the number in the middle of a series.\\
\textbf{Example:}\\
$2, 4, 6, 8, 10$ \\
The median here is 6, but if the number of elements in the series is even, we sum the two middle numbers and divide by 2.\\
\textbf{Example:}\\
$2, 4, 6, 8, 10, 12$ \\
The median here is:
\[
\frac{6 + 8}{2} = \frac{14}{2} = 7
\]

\section{Mode}
The mode is the most frequently occurring number in the series, so it is the easiest one to quickly identify.\\
\textbf{Example:}\\
$2, 4, 5, 4, 4, 6, 9$ \\
The mode here is: 4.

Sometimes there is no mode in a series.\\
\textbf{Example:}\\
$2, 3, 4, 5, 6$ \\
In this series, there is no mode.

\begin{flushright}
Created by: Saad Almalki \\
Using: \LaTeX
\end{flushright}

\end{document}
