\documentclass[12pt]{article}

\usepackage{amsmath}  % For mathematical symbols and equations
\usepackage{amssymb}  % For additional math symbols
\usepackage{geometry} % To control page margins
\geometry{a4paper, margin=1in}

\usepackage{hyperref} % To include hyperlinks
\usepackage{fancyhdr} % To customize the header and footer
\pagestyle{fancy}
\fancyhf{}
\fancyhead[L]{Probability and Statistics} 
\fancyfoot[C]{\thepage}

\title{Measures of Variability}
\author{Saad Almalki}
\date{\today}

\begin{document}

\maketitle

\section{Introduction} 
Measures of Variability, like central tendency, are important concepts in probability and statistics in math.

Some of these measures of variability are: the variance, standard deviation, and range.

\section{Range} 
The range is one of the important Measures of Variability. The range is simple to calculate: it is the maximum number minus the minimum number.\\
\textbf{Example:}\
$1, 2, 3, 4, 5$\\
The range is: 
\[ 5 - 1 = 4 \]

\section{Variance}
The variance is a measure of how data points differ from the mean.\\
\textbf{Example:}\
$2, 4, 6, 8, 10$\\
The mean here is:
\[
\frac{2 + 4 + 6 + 8 + 10}{5} = \frac{30}{5} = 6
\]
The variance is calculated as:
\[
\frac{(2-6)^2 + (4-6)^2 + (6-6)^2 + (8-6)^2 + (10-6)^2}{5} = \frac{16 + 4 + 0 + 4 + 16}{5} = \frac{40}{5} = 8
\]

\section{Standard Deviation}
The standard deviation is the square root of the variance and provides a measure of the spread of data around the mean.\\
\textbf{Example:}\
$2, 4, 6, 8, 10$\\
The variance is 8 (as calculated previously).\\
The standard deviation is:
\[
\sqrt{8} \approx 2.83
\]

\begin{flushright}
Created by: Saad Almalki \\
Using: \LaTeX
\end{flushright}

\end{document}
