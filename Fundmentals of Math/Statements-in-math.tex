\documentclass[12pt]{article}

\usepackage{amsmath}  % For mathematical symbols and equations
\usepackage{amssymb}  % For additional math symbols
\usepackage{geometry} % To control page margins
\geometry{a4paper, margin=1in}

\usepackage{hyperref} % To include hyperlinks
\usepackage{fancyhdr} % To customize the header and footer
\pagestyle{fancy}
\fancyhf{}
\fancyhead[L]{Mathematics Basics}
\fancyfoot[C]{\thepage}

\title{Statements in Mathematics}
\author{Saad Almalki}
\date{\today}

\begin{document}

\maketitle

\section{Statements in Mathematics}

A statement in mathematics is a declarative sentence that must be either true or false, but not both. Statements may also be referred to as **propositions**.

The important aspect of a statement is that it can be objectively answered with "yes" or "no," regardless of whether it is correct. Even if the result is false, the sentence itself qualifies as a statement because it can be evaluated.

Here are five examples to illustrate this concept:

\subsection{Examples}

\begin{enumerate}
    \item \(1 - 3 = 5\): This is a statement, but it is incorrect because the answer is clearly "no."
    \item "37": This is \textbf{not} a statement. If someone says this to you randomly on the street, your response might be, "What are you talking about?"
    \item "What is the answer to \(2 + 2\) = ?": This is not a statement because it is a question.
    \item "The answer to \(2 + 2 = 4\)": This is a correct statement, because you can say "yes, that is correct."
    \item "What is your name?": This is not a statement, as it requires an answer other than "true" or "false."
\end{enumerate}

\section{Conclusion}

This basic document about introduction to statements in mathematics. A statement must always have a definite truth value — either true or false — but never both. Understanding statements is very important for logic and mathematics.

\begin{flushright}
Created by: Saad Almalki \\
Using: \LaTeX
\end{flushright}

\end{document}
