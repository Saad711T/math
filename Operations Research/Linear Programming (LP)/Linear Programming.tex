\documentclass[12pt]{article}

\usepackage{amsmath}  % For mathematical symbols and equations
\usepackage{amssymb}  % For additional math symbols
\usepackage{geometry} % To control page margins
\geometry{a4paper, margin=1in}

\usepackage{hyperref} % To include hyperlinks
\usepackage{fancyhdr} % To customize the header and footer
\pagestyle{fancy}
\fancyhf{}
\fancyhead[L]{Linear Programming} 
\fancyfoot[C]{\thepage}

\title{Linear Programming}
\author{Saad Almalki}
\date{\today}

\begin{document}

\maketitle

\section{Introduction} 
Linear programming (LP) is a mathematical technique for optimizing a linear objective function, subject to linear constraints. It is widely used in operations research and economics.

\section{Graphical Method} 
The graphical method is used to solve linear programming problems with two decision variables by representing the constraints and objective function graphically.

\textbf{Example:} Maximize \( Z = 3x + 2y \) subject to:
\[
  x + 2y \leq 8
\]
\[
  3x + 2y \leq 12
\]
\[
  x, y \geq 0
\]

\subsection{Steps:}
1. Convert inequalities into equations.
2. Plot constraint lines on a graph.
3. Identify the feasible region.
4. Determine the optimal point by evaluating the objective function at corner points.

\section{Simplex Method}
The simplex method is an algebraic approach used to solve LP problems with more than two decision variables efficiently , In simplex we using "Simplex Tableau" to find "Optimal solution" .

\textbf{Example:} Maximize \( Z = 4x_1 + 3x_2 \) subject to:
\[
  2x_1 + x_2 \leq 8
\]
\[
  x_1 + 2x_2 \leq 6
\]
\[
  x_1, x_2 \geq 0
\]

\subsection{Steps:}
1. Convert inequalities into equations by introducing slack variables.
2. Construct the initial simplex tableau.
3. Identify the entering and leaving variables.
4. Perform pivoting to improve the objective function.
5. Repeat the process until an optimal solution is reached.

\begin{flushright}
Created by: Saad Almalki \\
Using: \LaTeX
\end{flushright}

\end{document}
