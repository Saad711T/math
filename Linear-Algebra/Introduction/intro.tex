\documentclass[12pt]{article}





\usepackage{amsmath}  % For mathematical symbols and equations
\usepackage{amssymb}  % For additional math symbols
\usepackage{geometry} % To control page margins
\geometry{a4paper, margin=1in}





\usepackage{hyperref} % To include hyperlinks
\usepackage{fancyhdr} % To customize the header and footer
\pagestyle{fancy}
\fancyhf{}
\fancyhead[L]{Introduction to Linear Algebra} 
\fancyfoot[C]{\thepage}





\title{Introduction to Linear Algebra}
\author{Saad Almalki}
\date{\today}

\begin{document}

\maketitle

\section{Section 1} 
Linear algebra is a branch of mathematics that studies vectors, vector spaces, linear transformations, and systems of linear equations. It is fundamental to many areas of mathematics and has applications in physics, engineering, computer science, and economics.

\subsection{Methods:}
1. Gaussian Elimination
2. Matrix Inversion
3. Eigenvalue Decomposition


\section{Summary}
In summary, linear algebra provides the tools and techniques to analyze and solve problems involving linear systems, vector spaces, and linear transformations. It is essential for understanding the structure of mathematical objects and has wide-ranging applications in various fields.

\begin{flushright}
Created by: Saad Almalki \\
\end{flushright}

\end{document}
